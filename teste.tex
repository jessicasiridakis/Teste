\documentclass{article}
\usepackage[utf8]{inputenc}
\usepackage{amsmath}
\author{Jéssica de Souza Pereira}
\date{15/04/2013}
\title{Como cortar o cabelo do seu namorado enquanto ele estiver dormindo.}
\newcommand{\vetor}[1]{\textbf{#1}}
\begin{document}

\maketitle
\newpage
\section{Introdução}
\label{sec:intro}
Neste trabalho iremos apresentar novas técnicas para abordagem de seu namorado enquanto este estiver adormecido para cortar seu cabelo cuidadosamente sem ser pega no flagra.
%aqui eu tenho que fazer alguma coisa para comentar xD
\section{Motivação}
\label{sec:motiv}
O cabelo dele está muito grande e faz cócegas enquanto durmímos. O procedimento pode mudar a vida de algumas mulheres.

Como vimos na introdução {seção \ref{sec:intro}},...

Uma fórmula qualquer: $V = R*I$.

\hfill

Uma fórmula qualquer: $$I = \frac{V}{R}$$
$$0=1+e^{j\pi}$$
IDFT (fórmula número \ref{eq:idft}):
\begin{equation}
\label{eq:idft}
x_n = \frac{1}{N}\sum_{k = 0}^{N-1} X_k \cdot e^{i2\pi kn/N}
\end{equation}
Equações de Maxwell (\ref{eq:max1} e \ref{eq:max2}).
\begin{align}
\label{eq:max1}
\nabla \cdot \vetor{E} & = \frac{\rho}{\epsilon_0} \\
\label{eq:max2}
\nabla \cdot B & = 0 \\
\label{eq:max3}
\nabla \times \vetor{E} & = \frac{\partial B}{\partial t}\\
\label{eq:max4}
\nabla \times \vetor{B} & = \mu_0\vetor{J} + \mu_0 \epsilon_0 + \frac{\partial \vetor{E}}{\partial t}
\end{align}


\end{document}